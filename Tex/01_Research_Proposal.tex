% !TeX root = ../Tex/main.tex

\section{Introduction}
In this article, I aim to examine the demographic dynamics of Brazil, with a particular focus on gender-related policies.
The Brazilian case offers a compelling context for analysis, as the evolution of gender and LGBTQ+ rights in the country has followed a trajectory distinct from that of Europe. Both Brazilian legislation and culture have been at the forefront of progressive reforms, partly due to historical and political circumstances. Notably, LGBTQ+ movements played a significant role in supporting the current Constitution, which has protected such rights since 1988. Knowledge and insights about the historical context and institutional settings are drawn from Encyclopedia Britannica and \cite{southey_history_2012}. 
\vspace{1em}

The author aims to study the impact of the institutional settings on demographic and social variables. In particular, emphasis will be on the effects of marriage and education laws on variables such as fertility and emancipation of women. The author will use the introduction of different laws and use them to assess the impact on variables that describe the marital and educational status of women.\vspace{1em}


\cite{alves_context_2017} identified two main sources of change through a qualitative analysis. Following in his footsteps, the article will mainly focus on two laws: the "National Policy on Sexual and Reproductive Rights in 2005" and the "National Family Planning Policy in 2007". These norms will serve as instruments to predict a changes in emancipation of women. They are expected to positively affect the outcome variables.
\vspace{1em}


A preliminary analysis of the literature on education in developing countries, highlighted a study from \cite{turmena_reforma_2022}. 
The journal article constitutes the milestone of this article and serves as main reference for the literature on education in Brazil.
This paper suggests that the "Law No. 5.692/1971 (Reforma do Ensino de 1º e 2º Graus)" had a major effects on education.
I will exploit data related to this law. The analysis I intend to conduct will depend on the availability of data.
\vspace{1em}

Potential data sources include IPUMS and the Instituto Brasileiro de Geografia e Estatística (IBGE). Additionally, aggregated data may be retrieved from the Brazilian Education Panel Databases, which covers the period from 1996 to 2015. Eventually, the article from \cite{rubiane_daniele_cardoso_de_almeida_demographic_2023} offers panel data on some demographic aspects.
\vspace{1em}

\textit{To establish causality} between the dependent and independent variables, the analysis will employ causal inference techniques. As a possible solution to the identification problem the study will provide results from a Difference-in-Difference. This technique is able to isolate the effect produced by the introduction of the laws in the  variables that approximate fertility or women emancipation.
Nonetheless, without appropriate control variables no identification strategy is reliable. Following best practices of the political science field, only a deep analysis of the literature will provide for suitable control variables.