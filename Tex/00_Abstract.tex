% !TeX root = ../Tex/main.tex

\begin{center}
    \subsection*{Abstract}
\end{center}
<<<<<<< HEAD
This research proposal investigates the determinants of educational outcomes in Brazil thanks to a causal-inference framework. Motivated by the persistent empirical uncertainty surrounding what works in education, the study exploits Brazil's federal structure and rich subnational variation to estimate the effect of school inputs on student performance and progression.  
=======
This research proposal investigates the determinants of educational outcomes in Brazil thsnks to a causal-inference framework. Motivated by the persistent empirical uncertainty surrounding what works in education, the study exploits Brazil's federal structure and rich subnational variation to estimate the effect of school inputs on student performance and progression.  
>>>>>>> 217951c006743ebc4550e4cb6285e2383704a500

\vspace*{1em}
Empirically, the project relies on the Brazilian Education Panel Databases \parencite{hubertsBrazilianEducationPanel2025}. The analysis builds on a causal directed acyclic graph (DAG) and a two-margin conceptualization of outcomes. Teacher quality is operationalized using measurable proxies--teachers' education and class size (students per class)--and is complemented by indicators of school infrastructure and a set of controls.

\vspace*{1em}
Identification proceeds in two steps. First, fixed-effects panel regressions with municipality, state, and year effects absorb time-invariant local heterogeneity, broader state-level differences, and common temporal shocks. Second, a Difference-in-Differences design leverages institutional and policy variation in education governance to sharpen causal interpretation. The project tests whether improvements in teacher quality raise median achievement and disproportionately benefit lower-performing students, while also examining whether observed score gains align with changes in failure and progression. By jointly analyzing these margins, the study aims to provide policy-relevant evidence on which educational investments most plausibly translate into genuine learning gains rather than shifts in selection or promotion dynamics.

% \lipsum[1] % genera il primo paragrafo di testo Lorem Ipsum