% !TeX root = ../Tex/main.tex
\section{Limitations}
From the very first page, in this paper I adopted a transparet approach. Therefore, this section highlights all limitation of the analysis conducted so far. In this way, constructive criticism become active part of the scientific process.

I noticed three weakeness of the identification strategy:
\begin{itemize}
    \item[-] Omitted Variable Bias.
    \item[-] Provide tests for observable implications in as many as possible narrow, focused, controlled circumstances.
    \item[-] Do not include control variables that are a consequence of the key IV.
\end{itemize}

Omitted Variable Bias is a common burden of all empirical scientists. There is no way to include all possible sources of variation in the identification strategy and hampers the possibility of identifiyng the "perfect" causal mechanism. However, there are painkiller to this issue. They do not come from causal inference or econometrics. Instead, they come from institutional and qualitative knowledge of the topic under study.

Thus, we can reassure the reader of the reliability of the empirical result deepening the qualitative knowledge of the problem at stake.

\vspace*{1em}
A second important limit of this study comes from the fact that the more we generilise the results, the less we can be sure of what we state. Empirical tests have increadible internal validity. Nonetheless, they all suffer of a noticeable restriction, namely external validity. The identification strategy is solid only when we test the causal mechanism in data it was though to work for. When we appkly the identification strategy to external data, coming from a similar data generating process, the rusults tend to become fuzzy, and the ground of the analysis becomes slippery. 

\vspace*{1em}
Eventually, we have another source of doubts about the research proposal.
Let's imagin a state investing more in school with better tests scores or with lower failure rates. In this way, school managers or regional governments are forced to act in accordance with an incentives scheme.

However, if this were the case for Brazil, there would be a huge bais in the identification strategy used so far. This bias is due to the fact that the outcome variable (test score for example) shapes investments in the school infrastracture, that at the same time influences students performances.

Further exploration of the institutioinal setting can shed light on this issue.

\section*{Conclusion}


%------- Acknowledgements -------
\subsubsection*{Acknowledgements}
Artificial intelligence-based tools were employed solely to improve linguistic clarity and grammar. No AI system contributed to the development of the research questions, theoretical framework or conclusions presented in this paper.