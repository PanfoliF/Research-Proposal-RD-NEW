% !TeX root = ../Tex/main.tex

\section*{Introduction}
This paper proposes a research project on educational data. The aim is establish causality between variables of interest using state of the art causal inferene techniques. The Brazilian context offers interesting sources of variation among data as well as granular observations. These are key factors in providing rubust empirical analysis.

\textcolor{red}{Mancano}
\begin{itemize}
    \item Unit of analysis
    \item research question
    \item hypothesis
\end{itemize}
Tutte e tre vanno bene evidenziate per il prof

Motivation: under many aspects there is little knowledge of what causes good educational outcomes \textcolor{red}{citation}. Many factors concur in formig students, thus scholars have always had hard times in distinguishing chains of causes and effects. The proposed paper could provide some more evidence.

Data: such a high aim requires the best tools. The author put a lot of effort in the data collection step, many institutional websites have been surfed until the best source of data come out. The Brazilian Education Panel Databases \parencite{hubertsBrazilianEducationPanel2025}, appears the best source of observation for the proposed task. Not only it provides researchers with granular data, it also comes from a reliable and well know institution, namely IDB (Interamerican Development Bank).

Methodology: If we are to establish causality in a credible way, we need to use causal inference methodologies. The author will provide the reader with extensive axplantion of emplyed identification strategy in the methodology section (\nameref{sub:methodology}).The following lines briefly anticipate that section: The identification strategy selected for our aim is splet in two: first, due controls come along with two regressions. Secondly, a Difference in Difference application of these regressions, should isolate with more efficacy the causal effec of independent variables on dependent one.
An appropriate review of the literature provides for suitable control variables.

My goal is to distinguish two channels: standardized test performance (\textit{extensive margin}) and grade progression outcomes (\textit{intensive margin}).

Policy value: The following pages are intended to create value for policy makers. Putting aside cumbersome vocabulary and contex related dictions, they should inform policy making in the field of education in a simple and transparent manner.
The results of the analysis may open new ways of investing in education and shed light on wiser policy aims.

% Andrew Abbott Methods of Discovery, 2004
% Apply existing theory to different data or cases

% Sieberer
% Do not include control variables that are a consequence of the key IV.

% Alternatives to Statistical Control
% Substitute research design for control variables (i.e. control through case selection)
% – Test a theory in a particular place, at a particular time.
% – Provide tests for observable implications in as many as possible narrow, focused, controlled circumstances. e.g. particular actors, groups, space, time, natural experiments…
% – Choose circumstances where competing theories provide deviant observable implications.